% Options for packages loaded elsewhere
\PassOptionsToPackage{unicode}{hyperref}
\PassOptionsToPackage{hyphens}{url}
\PassOptionsToPackage{dvipsnames,svgnames,x11names}{xcolor}
%
\documentclass[
  ignorenonframetext,
]{beamer}
\usepackage{pgfpages}
\setbeamertemplate{caption}[numbered]
\setbeamertemplate{caption label separator}{: }
\setbeamercolor{caption name}{fg=normal text.fg}
\beamertemplatenavigationsymbolsempty
% Prevent slide breaks in the middle of a paragraph
\widowpenalties 1 10000
\raggedbottom

\usepackage{amsmath,amssymb}
\usepackage{iftex}
\ifPDFTeX
  \usepackage[T1]{fontenc}
  \usepackage[utf8]{inputenc}
  \usepackage{textcomp} % provide euro and other symbols
\else % if luatex or xetex
  \usepackage{unicode-math}
  \defaultfontfeatures{Scale=MatchLowercase}
  \defaultfontfeatures[\rmfamily]{Ligatures=TeX,Scale=1}
\fi
\usepackage{lmodern}
\usetheme[]{AnnArbor}
\usecolortheme{dolphin}
\usefonttheme{structurebold}
\ifPDFTeX\else  
    % xetex/luatex font selection
\fi
% Use upquote if available, for straight quotes in verbatim environments
\IfFileExists{upquote.sty}{\usepackage{upquote}}{}
\IfFileExists{microtype.sty}{% use microtype if available
  \usepackage[]{microtype}
  \UseMicrotypeSet[protrusion]{basicmath} % disable protrusion for tt fonts
}{}
\makeatletter
\@ifundefined{KOMAClassName}{% if non-KOMA class
  \IfFileExists{parskip.sty}{%
    \usepackage{parskip}
  }{% else
    \setlength{\parindent}{0pt}
    \setlength{\parskip}{6pt plus 2pt minus 1pt}}
}{% if KOMA class
  \KOMAoptions{parskip=half}}
\makeatother
\usepackage{xcolor}
\newif\ifbibliography
\setlength{\emergencystretch}{3em} % prevent overfull lines
\setcounter{secnumdepth}{-\maxdimen} % remove section numbering

\usepackage{color}
\usepackage{fancyvrb}
\newcommand{\VerbBar}{|}
\newcommand{\VERB}{\Verb[commandchars=\\\{\}]}
\DefineVerbatimEnvironment{Highlighting}{Verbatim}{commandchars=\\\{\}}
% Add ',fontsize=\small' for more characters per line
\usepackage{framed}
\definecolor{shadecolor}{RGB}{241,243,245}
\newenvironment{Shaded}{\begin{snugshade}}{\end{snugshade}}
\newcommand{\AlertTok}[1]{\textcolor[rgb]{0.68,0.00,0.00}{#1}}
\newcommand{\AnnotationTok}[1]{\textcolor[rgb]{0.37,0.37,0.37}{#1}}
\newcommand{\AttributeTok}[1]{\textcolor[rgb]{0.40,0.45,0.13}{#1}}
\newcommand{\BaseNTok}[1]{\textcolor[rgb]{0.68,0.00,0.00}{#1}}
\newcommand{\BuiltInTok}[1]{\textcolor[rgb]{0.00,0.23,0.31}{#1}}
\newcommand{\CharTok}[1]{\textcolor[rgb]{0.13,0.47,0.30}{#1}}
\newcommand{\CommentTok}[1]{\textcolor[rgb]{0.37,0.37,0.37}{#1}}
\newcommand{\CommentVarTok}[1]{\textcolor[rgb]{0.37,0.37,0.37}{\textit{#1}}}
\newcommand{\ConstantTok}[1]{\textcolor[rgb]{0.56,0.35,0.01}{#1}}
\newcommand{\ControlFlowTok}[1]{\textcolor[rgb]{0.00,0.23,0.31}{\textbf{#1}}}
\newcommand{\DataTypeTok}[1]{\textcolor[rgb]{0.68,0.00,0.00}{#1}}
\newcommand{\DecValTok}[1]{\textcolor[rgb]{0.68,0.00,0.00}{#1}}
\newcommand{\DocumentationTok}[1]{\textcolor[rgb]{0.37,0.37,0.37}{\textit{#1}}}
\newcommand{\ErrorTok}[1]{\textcolor[rgb]{0.68,0.00,0.00}{#1}}
\newcommand{\ExtensionTok}[1]{\textcolor[rgb]{0.00,0.23,0.31}{#1}}
\newcommand{\FloatTok}[1]{\textcolor[rgb]{0.68,0.00,0.00}{#1}}
\newcommand{\FunctionTok}[1]{\textcolor[rgb]{0.28,0.35,0.67}{#1}}
\newcommand{\ImportTok}[1]{\textcolor[rgb]{0.00,0.46,0.62}{#1}}
\newcommand{\InformationTok}[1]{\textcolor[rgb]{0.37,0.37,0.37}{#1}}
\newcommand{\KeywordTok}[1]{\textcolor[rgb]{0.00,0.23,0.31}{\textbf{#1}}}
\newcommand{\NormalTok}[1]{\textcolor[rgb]{0.00,0.23,0.31}{#1}}
\newcommand{\OperatorTok}[1]{\textcolor[rgb]{0.37,0.37,0.37}{#1}}
\newcommand{\OtherTok}[1]{\textcolor[rgb]{0.00,0.23,0.31}{#1}}
\newcommand{\PreprocessorTok}[1]{\textcolor[rgb]{0.68,0.00,0.00}{#1}}
\newcommand{\RegionMarkerTok}[1]{\textcolor[rgb]{0.00,0.23,0.31}{#1}}
\newcommand{\SpecialCharTok}[1]{\textcolor[rgb]{0.37,0.37,0.37}{#1}}
\newcommand{\SpecialStringTok}[1]{\textcolor[rgb]{0.13,0.47,0.30}{#1}}
\newcommand{\StringTok}[1]{\textcolor[rgb]{0.13,0.47,0.30}{#1}}
\newcommand{\VariableTok}[1]{\textcolor[rgb]{0.07,0.07,0.07}{#1}}
\newcommand{\VerbatimStringTok}[1]{\textcolor[rgb]{0.13,0.47,0.30}{#1}}
\newcommand{\WarningTok}[1]{\textcolor[rgb]{0.37,0.37,0.37}{\textit{#1}}}

\providecommand{\tightlist}{%
  \setlength{\itemsep}{0pt}\setlength{\parskip}{0pt}}\usepackage{longtable,booktabs,array}
\usepackage{calc} % for calculating minipage widths
\usepackage{caption}
% Make caption package work with longtable
\makeatletter
\def\fnum@table{\tablename~\thetable}
\makeatother
\usepackage{graphicx}
\makeatletter
\def\maxwidth{\ifdim\Gin@nat@width>\linewidth\linewidth\else\Gin@nat@width\fi}
\def\maxheight{\ifdim\Gin@nat@height>\textheight\textheight\else\Gin@nat@height\fi}
\makeatother
% Scale images if necessary, so that they will not overflow the page
% margins by default, and it is still possible to overwrite the defaults
% using explicit options in \includegraphics[width, height, ...]{}
\setkeys{Gin}{width=\maxwidth,height=\maxheight,keepaspectratio}
% Set default figure placement to htbp
\makeatletter
\def\fps@figure{htbp}
\makeatother
% definitions for citeproc citations
\NewDocumentCommand\citeproctext{}{}
\NewDocumentCommand\citeproc{mm}{%
  \begingroup\def\citeproctext{#2}\cite{#1}\endgroup}
\makeatletter
 % allow citations to break across lines
 \let\@cite@ofmt\@firstofone
 % avoid brackets around text for \cite:
 \def\@biblabel#1{}
 \def\@cite#1#2{{#1\if@tempswa , #2\fi}}
\makeatother
\newlength{\cslhangindent}
\setlength{\cslhangindent}{1.5em}
\newlength{\csllabelwidth}
\setlength{\csllabelwidth}{3em}
\newenvironment{CSLReferences}[2] % #1 hanging-indent, #2 entry-spacing
 {\begin{list}{}{%
  \setlength{\itemindent}{0pt}
  \setlength{\leftmargin}{0pt}
  \setlength{\parsep}{0pt}
  % turn on hanging indent if param 1 is 1
  \ifodd #1
   \setlength{\leftmargin}{\cslhangindent}
   \setlength{\itemindent}{-1\cslhangindent}
  \fi
  % set entry spacing
  \setlength{\itemsep}{#2\baselineskip}}}
 {\end{list}}
\usepackage{calc}
\newcommand{\CSLBlock}[1]{\hfill\break\parbox[t]{\linewidth}{\strut\ignorespaces#1\strut}}
\newcommand{\CSLLeftMargin}[1]{\parbox[t]{\csllabelwidth}{\strut#1\strut}}
\newcommand{\CSLRightInline}[1]{\parbox[t]{\linewidth - \csllabelwidth}{\strut#1\strut}}
\newcommand{\CSLIndent}[1]{\hspace{\cslhangindent}#1}


% logo
\titlegraphic{\includegraphics[width=4cm]{../000_logos/logo-blue-vertical}}
\logo{\ifnum\thepage>1\includegraphics[width=0.5cm]{../000_logos/logo-blue-vertical}\fi}

% UMNG: Manual de image institucional

% Colors

% Umng
\definecolor{yellow}{HTML}{fdc600}
\definecolor{red}{HTML}{ee2a24}

% Estudios a Distancia
\definecolor{blue1}{HTML}{12245b}
\definecolor{blue2}{HTML}{767ca6}
\definecolor{blue3}{HTML}{cad2ec}

% Modify items
\setbeamercolor{palette primary}{bg=blue3}
\setbeamercolor{palette tertiary}{bg=blue1}
\setbeamercolor{frametitle}{bg=yellow}

% Hyperlinks
\hypersetup{
  linkcolor=red,
  citecolor=red
}

\makeatletter
\@ifpackageloaded{caption}{}{\usepackage{caption}}
\AtBeginDocument{%
\ifdefined\contentsname
  \renewcommand*\contentsname{Table of contents}
\else
  \newcommand\contentsname{Table of contents}
\fi
\ifdefined\listfigurename
  \renewcommand*\listfigurename{List of Figures}
\else
  \newcommand\listfigurename{List of Figures}
\fi
\ifdefined\listtablename
  \renewcommand*\listtablename{List of Tables}
\else
  \newcommand\listtablename{List of Tables}
\fi
\ifdefined\figurename
  \renewcommand*\figurename{Figure}
\else
  \newcommand\figurename{Figure}
\fi
\ifdefined\tablename
  \renewcommand*\tablename{Table}
\else
  \newcommand\tablename{Table}
\fi
}
\@ifpackageloaded{float}{}{\usepackage{float}}
\floatstyle{ruled}
\@ifundefined{c@chapter}{\newfloat{codelisting}{h}{lop}}{\newfloat{codelisting}{h}{lop}[chapter]}
\floatname{codelisting}{Listing}
\newcommand*\listoflistings{\listof{codelisting}{List of Listings}}
\makeatother
\makeatletter
\makeatother
\makeatletter
\@ifpackageloaded{caption}{}{\usepackage{caption}}
\@ifpackageloaded{subcaption}{}{\usepackage{subcaption}}
\makeatother

\ifLuaTeX
\usepackage[bidi=basic]{babel}
\else
\usepackage[bidi=default]{babel}
\fi
\babelprovide[main,import]{english}
% get rid of language-specific shorthands (see #6817):
\let\LanguageShortHands\languageshorthands
\def\languageshorthands#1{}
\ifLuaTeX
  \usepackage{selnolig}  % disable illegal ligatures
\fi
\usepackage{bookmark}

\IfFileExists{xurl.sty}{\usepackage{xurl}}{} % add URL line breaks if available
\urlstyle{same} % disable monospaced font for URLs
\hypersetup{
  pdftitle={An Overview of the R Language},
  pdfauthor={Luis Francisco Gómez López},
  pdflang={en},
  colorlinks=true,
  linkcolor={Maroon},
  filecolor={Maroon},
  citecolor={Blue},
  urlcolor={Blue},
  pdfcreator={LaTeX via pandoc}}


\title{An Overview of the R Language}
\author{Luis Francisco Gómez López}
\date{2024-08-01}
\institute{FAEDIS}

\begin{document}
\frame{\titlepage}

\renewcommand*\contentsname{Table of contents}
\begin{frame}[allowframebreaks]
  \frametitle{Table of contents}
  \tableofcontents[hideallsubsections]
\end{frame}

\section{Please Read Me}\label{please-read-me}

\begin{frame}{}
\phantomsection\label{section}
\begin{itemize}
\tightlist
\item
  This presentation is based on (\citeproc{ref-chapman_r_2019}{Chapman
  and Feit 2019, chap. 2})
\end{itemize}
\end{frame}

\section{Purpose}\label{purpose}

\begin{frame}{}
\phantomsection\label{section-1}
\begin{itemize}
\tightlist
\item
  Equip beginners with a programming foundation by leveraging the R
  language, enabling practical application in marketing research and
  analytics
\end{itemize}
\end{frame}

\section{Sales and product satisfaction
survey}\label{sales-and-product-satisfaction-survey}

\begin{frame}{}
\phantomsection\label{section-2}
\begin{itemize}
\item
  \textbf{Ordinal 7 point scale}

  \begin{itemize}
  \tightlist
  \item
    Extremely satisfied: 7
  \item
    Moderately satisfied: 6
  \item
    Slightly satisfied: 5
  \item
    Neither satisfied or dissatisfied: 4
  \item
    Slightly dissatisfied: 3
  \item
    Moderately dissatisfied: 2
  \item
    Extremely dissatisfied: 1
  \end{itemize}
\item
  \textbf{Variables}

  \begin{itemize}
  \tightlist
  \item
    iProdSAT: satisfaction with a product
  \item
    iSalesSAT: satisfaction with sales experience
  \item
    iProdREC: likelihood to recommend the product
  \item
    iSalesREC: likelihood to recommend the sales person
  \item
    Segment: market segment assigned by a clustering algorithm
    (\citeproc{ref-chapman_r_2019}{Chapman and Feit 2019, chap. 11})
  \end{itemize}
\end{itemize}
\end{frame}

\begin{frame}[fragile]{}
\phantomsection\label{section-3}
\begin{itemize}
\tightlist
\item
  \textbf{Import data: the base R way}
\end{itemize}

\tiny

\begin{Shaded}
\begin{Highlighting}[]
\NormalTok{satisfaction\_data }\OtherTok{\textless{}{-}} \FunctionTok{read.csv}\NormalTok{(}\AttributeTok{file =} \StringTok{"http://goo.gl/UDv12g"}\NormalTok{)}
\NormalTok{satisfaction\_data }\SpecialCharTok{|\textgreater{}} \FunctionTok{head}\NormalTok{(}\AttributeTok{n=}\DecValTok{5}\NormalTok{)}
\end{Highlighting}
\end{Shaded}

\begin{verbatim}
  iProdSAT iSalesSAT Segment iProdREC iSalesREC
1        6         2       1        4         3
2        4         5       3        4         4
3        5         3       4        5         4
4        3         3       2        4         4
5        3         3       3        2         2
\end{verbatim}
\end{frame}

\begin{frame}[fragile]{}
\phantomsection\label{section-4}
\begin{itemize}
\tightlist
\item
  \textbf{Import data: the tidyverse way}
\end{itemize}

\tiny

\begin{Shaded}
\begin{Highlighting}[]
\FunctionTok{library}\NormalTok{(tidyverse) }\CommentTok{\# Remember to load the tidyverse library}
\NormalTok{satisfaction\_data }\OtherTok{\textless{}{-}} \FunctionTok{read\_csv}\NormalTok{(}\AttributeTok{file =} \StringTok{"http://goo.gl/UDv12g"}\NormalTok{)}
\NormalTok{satisfaction\_data }\SpecialCharTok{|\textgreater{}} \FunctionTok{head}\NormalTok{(}\AttributeTok{n=}\DecValTok{5}\NormalTok{)}
\end{Highlighting}
\end{Shaded}

\begin{verbatim}
# A tibble: 5 x 5
  iProdSAT iSalesSAT Segment iProdREC iSalesREC
     <dbl>     <dbl>   <dbl>    <dbl>     <dbl>
1        6         2       1        4         3
2        4         5       3        4         4
3        5         3       4        5         4
4        3         3       2        4         4
5        3         3       3        2         2
\end{verbatim}
\end{frame}

\begin{frame}[fragile]{}
\phantomsection\label{section-5}
\begin{itemize}
\tightlist
\item
  \textbf{Transform data: the base R way}
\end{itemize}

\tiny

\begin{Shaded}
\begin{Highlighting}[]
\NormalTok{satisfaction\_data}\SpecialCharTok{$}\NormalTok{Segment }\OtherTok{\textless{}{-}} \FunctionTok{factor}\NormalTok{(}\AttributeTok{x =}\NormalTok{ satisfaction\_data}\SpecialCharTok{$}\NormalTok{Segment, }
                                    \AttributeTok{ordered =} \ConstantTok{FALSE}\NormalTok{)}
\FunctionTok{summary}\NormalTok{(}\AttributeTok{object =}\NormalTok{ satisfaction\_data)}
\end{Highlighting}
\end{Shaded}

\begin{verbatim}
    iProdSAT      iSalesSAT     Segment    iProdREC       iSalesREC    
 Min.   :1.00   Min.   :1.000   1: 54   Min.   :1.000   Min.   :1.000  
 1st Qu.:3.00   1st Qu.:3.000   2:131   1st Qu.:3.000   1st Qu.:3.000  
 Median :4.00   Median :4.000   3:154   Median :4.000   Median :3.000  
 Mean   :4.13   Mean   :3.802   4:161   Mean   :4.044   Mean   :3.444  
 3rd Qu.:5.00   3rd Qu.:5.000           3rd Qu.:5.000   3rd Qu.:4.000  
 Max.   :7.00   Max.   :7.000           Max.   :7.000   Max.   :7.000  
\end{verbatim}
\end{frame}

\begin{frame}[fragile]{}
\phantomsection\label{section-6}
\begin{itemize}
\item
  \textbf{Transform data: the skimr and the tidyverse way}

  \begin{itemize}
  \tightlist
  \item
    Ups the table is really big!!! Try it in your console to see the
    complete table
  \end{itemize}
\end{itemize}

\tiny

\begin{Shaded}
\begin{Highlighting}[]
\FunctionTok{library}\NormalTok{(skimr) }\CommentTok{\# Remember to install the package if it is not installed}
\NormalTok{satisfaction\_data }\OtherTok{\textless{}{-}}\NormalTok{ satisfaction\_data }\SpecialCharTok{|\textgreater{}}
  \FunctionTok{mutate}\NormalTok{(}\AttributeTok{Segment =} \FunctionTok{factor}\NormalTok{(}\AttributeTok{x =}\NormalTok{ Segment, }\AttributeTok{ordered =} \ConstantTok{FALSE}\NormalTok{))}
\end{Highlighting}
\end{Shaded}

\begin{Shaded}
\begin{Highlighting}[]
\NormalTok{satisfaction\_data }\SpecialCharTok{|\textgreater{}} \FunctionTok{skim}\NormalTok{()}
\end{Highlighting}
\end{Shaded}
\end{frame}

\begin{frame}{}
\phantomsection\label{section-7}
\begin{itemize}
\item
  \textbf{R objects}: everything in R is an object (object-oriented).
  For now, we will only inspect a few selected objects:

  \begin{itemize}
  \item
    \textbf{Atomic vectors}\footnote<.->{In R the atomic vectors are
      logical, integer, double, numeric (which includes integer and
      double), character, complex and raw but for pedagogical purposes
      we are going to check later character, numeric includes integer
      and double and we are not going to use complex and raw}

    \begin{itemize}
    \tightlist
    \item
      Logical
    \item
      Integer
    \item
      Double
    \end{itemize}
  \item
    \textbf{Factors}
  \item
    \textbf{Data Frames}
  \item
    \textbf{Tibbles}
  \end{itemize}
\end{itemize}
\end{frame}

\begin{frame}[fragile]{}
\phantomsection\label{section-8}
\begin{itemize}
\item
  \textbf{Atomic vectors}

  \begin{itemize}
  \tightlist
  \item
    Logical
  \end{itemize}
\end{itemize}

\tiny

\begin{Shaded}
\begin{Highlighting}[]
\FunctionTok{as.integer}\NormalTok{(satisfaction\_data}\SpecialCharTok{$}\NormalTok{Segment)[}\DecValTok{1}\SpecialCharTok{:}\DecValTok{5}\NormalTok{] }\SpecialCharTok{==} \DecValTok{1}
\end{Highlighting}
\end{Shaded}

\begin{verbatim}
[1]  TRUE FALSE FALSE FALSE FALSE
\end{verbatim}

\begin{Shaded}
\begin{Highlighting}[]
\FunctionTok{as.integer}\NormalTok{(satisfaction\_data}\SpecialCharTok{$}\NormalTok{Segment)[}\DecValTok{1}\SpecialCharTok{:}\DecValTok{5}\NormalTok{] }\SpecialCharTok{\textgreater{}} \DecValTok{1}
\end{Highlighting}
\end{Shaded}

\begin{verbatim}
[1] FALSE  TRUE  TRUE  TRUE  TRUE
\end{verbatim}

\begin{Shaded}
\begin{Highlighting}[]
\FunctionTok{as.integer}\NormalTok{(satisfaction\_data}\SpecialCharTok{$}\NormalTok{Segment)[}\DecValTok{1}\SpecialCharTok{:}\DecValTok{5}\NormalTok{] }\SpecialCharTok{\textgreater{}=} \DecValTok{1}
\end{Highlighting}
\end{Shaded}

\begin{verbatim}
[1] TRUE TRUE TRUE TRUE TRUE
\end{verbatim}
\end{frame}

\begin{frame}[fragile]{}
\phantomsection\label{section-9}
\begin{itemize}
\item
  \textbf{Atomic vectors}

  \begin{itemize}
  \tightlist
  \item
    Integer
  \end{itemize}
\end{itemize}

\tiny

\begin{Shaded}
\begin{Highlighting}[]
\FunctionTok{as.integer}\NormalTok{(satisfaction\_data}\SpecialCharTok{$}\NormalTok{Segment)[}\DecValTok{1}\SpecialCharTok{:}\DecValTok{5}\NormalTok{]}
\end{Highlighting}
\end{Shaded}

\begin{verbatim}
[1] 1 3 4 2 3
\end{verbatim}

\normalsize

\begin{itemize}
\item
  \textbf{Atomic vectors}

  \begin{itemize}
  \tightlist
  \item
    Double
  \end{itemize}
\end{itemize}

\tiny

\begin{Shaded}
\begin{Highlighting}[]
\FunctionTok{sprintf}\NormalTok{(}\StringTok{"\%.2f"}\NormalTok{, satisfaction\_data}\SpecialCharTok{$}\NormalTok{iProdSAT[}\DecValTok{1}\SpecialCharTok{:}\DecValTok{5}\NormalTok{])}
\end{Highlighting}
\end{Shaded}

\begin{verbatim}
[1] "6.00" "4.00" "5.00" "3.00" "3.00"
\end{verbatim}
\end{frame}

\begin{frame}[fragile]{}
\phantomsection\label{section-10}
\begin{itemize}
\tightlist
\item
  \textbf{Factors}
\end{itemize}

\tiny

\begin{Shaded}
\begin{Highlighting}[]
\NormalTok{satisfaction\_data}\SpecialCharTok{$}\NormalTok{Segment[}\DecValTok{1}\SpecialCharTok{:}\DecValTok{5}\NormalTok{]}
\end{Highlighting}
\end{Shaded}

\begin{verbatim}
[1] 1 3 4 2 3
Levels: 1 2 3 4
\end{verbatim}

\normalsize

\begin{itemize}
\tightlist
\item
  \textbf{Data Frames}
\end{itemize}

\tiny

\begin{Shaded}
\begin{Highlighting}[]
\FunctionTok{as.data.frame}\NormalTok{(satisfaction\_data) }\SpecialCharTok{|\textgreater{}} \FunctionTok{head}\NormalTok{(}\AttributeTok{n=}\DecValTok{5}\NormalTok{)}
\end{Highlighting}
\end{Shaded}

\begin{verbatim}
  iProdSAT iSalesSAT Segment iProdREC iSalesREC
1        6         2       1        4         3
2        4         5       3        4         4
3        5         3       4        5         4
4        3         3       2        4         4
5        3         3       3        2         2
\end{verbatim}
\end{frame}

\begin{frame}[fragile]{}
\phantomsection\label{section-11}
\begin{itemize}
\tightlist
\item
  \textbf{Tibble}
\end{itemize}

\tiny

\begin{Shaded}
\begin{Highlighting}[]
\FunctionTok{class}\NormalTok{(satisfaction\_data)}
\end{Highlighting}
\end{Shaded}

\begin{verbatim}
[1] "tbl_df"     "tbl"        "data.frame"
\end{verbatim}

\begin{Shaded}
\begin{Highlighting}[]
\NormalTok{satisfaction\_data }\SpecialCharTok{|\textgreater{}} \FunctionTok{head}\NormalTok{(}\AttributeTok{n=}\DecValTok{5}\NormalTok{)}
\end{Highlighting}
\end{Shaded}

\begin{verbatim}
# A tibble: 5 x 5
  iProdSAT iSalesSAT Segment iProdREC iSalesREC
     <dbl>     <dbl> <fct>      <dbl>     <dbl>
1        6         2 1              4         3
2        4         5 3              4         4
3        5         3 4              5         4
4        3         3 2              4         4
5        3         3 3              2         2
\end{verbatim}
\end{frame}

\begin{frame}[fragile]{}
\phantomsection\label{section-12}
\begin{itemize}
\tightlist
\item
  \textbf{Add new variables: the base R way}
\end{itemize}

\tiny

\begin{Shaded}
\begin{Highlighting}[]
\NormalTok{satisfaction\_data}\SpecialCharTok{$}\NormalTok{customer }\OtherTok{\textless{}{-}} \DecValTok{1}\SpecialCharTok{:}\FunctionTok{nrow}\NormalTok{(satisfaction\_data)}
\FunctionTok{as.data.frame}\NormalTok{(satisfaction\_data) }\SpecialCharTok{|\textgreater{}} \FunctionTok{head}\NormalTok{(}\AttributeTok{n=}\DecValTok{5}\NormalTok{)}
\end{Highlighting}
\end{Shaded}

\begin{verbatim}
  iProdSAT iSalesSAT Segment iProdREC iSalesREC customer
1        6         2       1        4         3        1
2        4         5       3        4         4        2
3        5         3       4        5         4        3
4        3         3       2        4         4        4
5        3         3       3        2         2        5
\end{verbatim}

\normalsize

\begin{itemize}
\tightlist
\item
  \textbf{Add new variables: the tidyverse way}
\end{itemize}

\tiny

\begin{Shaded}
\begin{Highlighting}[]
\NormalTok{satisfaction\_data }\SpecialCharTok{|\textgreater{}}
  \FunctionTok{mutate}\NormalTok{(}\AttributeTok{customer =} \DecValTok{1}\SpecialCharTok{:}\FunctionTok{nrow}\NormalTok{(satisfaction\_data)) }\SpecialCharTok{|\textgreater{}} 
  \FunctionTok{head}\NormalTok{(}\AttributeTok{n=}\DecValTok{5}\NormalTok{)}
\end{Highlighting}
\end{Shaded}

\begin{verbatim}
# A tibble: 5 x 6
  iProdSAT iSalesSAT Segment iProdREC iSalesREC customer
     <dbl>     <dbl> <fct>      <dbl>     <dbl>    <int>
1        6         2 1              4         3        1
2        4         5 3              4         4        2
3        5         3 4              5         4        3
4        3         3 2              4         4        4
5        3         3 3              2         2        5
\end{verbatim}
\end{frame}

\begin{frame}[fragile]{}
\phantomsection\label{section-13}
\begin{itemize}
\tightlist
\item
  \textbf{Picks variables based on their names: the base R way}
\end{itemize}

\tiny

\begin{Shaded}
\begin{Highlighting}[]
\FunctionTok{as.data.frame}\NormalTok{(satisfaction\_data)[}\FunctionTok{c}\NormalTok{(}\StringTok{"customer"}\NormalTok{, }\StringTok{"Segment"}\NormalTok{,}
                                   \StringTok{"iProdSAT"}\NormalTok{, }\StringTok{"iSalesSAT"}\NormalTok{, }\StringTok{"iProdREC"}\NormalTok{, }\StringTok{"iSalesREC"}\NormalTok{)] }\SpecialCharTok{|\textgreater{}}
  \FunctionTok{head}\NormalTok{(}\AttributeTok{n=}\DecValTok{5}\NormalTok{)}
\end{Highlighting}
\end{Shaded}

\begin{verbatim}
  customer Segment iProdSAT iSalesSAT iProdREC iSalesREC
1        1       1        6         2        4         3
2        2       3        4         5        4         4
3        3       4        5         3        5         4
4        4       2        3         3        4         4
5        5       3        3         3        2         2
\end{verbatim}

\normalsize

\begin{itemize}
\tightlist
\item
  \textbf{Picks variables based on their names: the tidyverse way}
\end{itemize}

\tiny

\begin{Shaded}
\begin{Highlighting}[]
\NormalTok{satisfaction\_data }\SpecialCharTok{|\textgreater{}}
  \FunctionTok{select}\NormalTok{(customer, Segment, iProdSAT, iSalesSAT, iProdREC, iSalesREC) }\SpecialCharTok{|\textgreater{}}
  \FunctionTok{head}\NormalTok{(}\AttributeTok{n=}\DecValTok{5}\NormalTok{)}
\end{Highlighting}
\end{Shaded}

\begin{verbatim}
# A tibble: 5 x 6
  customer Segment iProdSAT iSalesSAT iProdREC iSalesREC
     <int> <fct>      <dbl>     <dbl>    <dbl>     <dbl>
1        1 1              6         2        4         3
2        2 3              4         5        4         4
3        3 4              5         3        5         4
4        4 2              3         3        4         4
5        5 3              3         3        2         2
\end{verbatim}
\end{frame}

\begin{frame}[fragile]{}
\phantomsection\label{section-14}
\begin{itemize}
\tightlist
\item
  \textbf{Picks cases based on their values: the base R way}
\end{itemize}

\tiny

\begin{Shaded}
\begin{Highlighting}[]
\FunctionTok{as.data.frame}\NormalTok{(satisfaction\_data)[satisfaction\_data}\SpecialCharTok{$}\NormalTok{Segment }\SpecialCharTok{==} \DecValTok{2}\NormalTok{, ] }\SpecialCharTok{|\textgreater{}} 
  \FunctionTok{head}\NormalTok{(}\AttributeTok{n=}\DecValTok{5}\NormalTok{)}
\end{Highlighting}
\end{Shaded}

\begin{verbatim}
   iProdSAT iSalesSAT Segment iProdREC iSalesREC customer
4         3         3       2        4         4        4
14        4         3       2        3         2       14
18        3         5       2        3         3       18
19        4         4       2        1         1       19
23        4         2       2        4         6       23
\end{verbatim}

\normalsize

\begin{itemize}
\tightlist
\item
  \textbf{Picks cases based on their values: the tidyverse way}
\end{itemize}

\tiny

\begin{Shaded}
\begin{Highlighting}[]
\NormalTok{satisfaction\_data }\SpecialCharTok{|\textgreater{}}
  \FunctionTok{filter}\NormalTok{(Segment }\SpecialCharTok{==} \DecValTok{2}\NormalTok{) }\SpecialCharTok{|\textgreater{}}
  \FunctionTok{head}\NormalTok{(}\AttributeTok{n=}\DecValTok{5}\NormalTok{)}
\end{Highlighting}
\end{Shaded}

\begin{verbatim}
# A tibble: 5 x 6
  iProdSAT iSalesSAT Segment iProdREC iSalesREC customer
     <dbl>     <dbl> <fct>      <dbl>     <dbl>    <int>
1        3         3 2              4         4        4
2        4         3 2              3         2       14
3        3         5 2              3         3       18
4        4         4 2              1         1       19
5        4         2 2              4         6       23
\end{verbatim}
\end{frame}

\begin{frame}[fragile]{}
\phantomsection\label{section-15}
\begin{itemize}
\tightlist
\item
  \textbf{Reduces multiple values to a single summary: the base R way}
\end{itemize}

\tiny

\begin{Shaded}
\begin{Highlighting}[]
\FunctionTok{data.frame}\NormalTok{(}\AttributeTok{mean\_iProdSAT =} \FunctionTok{mean}\NormalTok{(satisfaction\_data}\SpecialCharTok{$}\NormalTok{iProdSAT),}
           \AttributeTok{median\_iSalesSAT =} \FunctionTok{median}\NormalTok{(satisfaction\_data}\SpecialCharTok{$}\NormalTok{iSalesSAT))}
\end{Highlighting}
\end{Shaded}

\begin{verbatim}
  mean_iProdSAT median_iSalesSAT
1          4.13                4
\end{verbatim}

\normalsize

\begin{itemize}
\tightlist
\item
  \textbf{Reduces multiple values to a single summary: the tidyverse
  way}
\end{itemize}

\tiny

\begin{Shaded}
\begin{Highlighting}[]
\NormalTok{satisfaction\_data }\SpecialCharTok{|\textgreater{}}
  \FunctionTok{summarise}\NormalTok{(}\AttributeTok{mean\_iProdSAT =} \FunctionTok{mean}\NormalTok{(iProdSAT), }\AttributeTok{median\_iSalesSAT =} \FunctionTok{median}\NormalTok{(iSalesSAT))}
\end{Highlighting}
\end{Shaded}

\begin{verbatim}
# A tibble: 1 x 2
  mean_iProdSAT median_iSalesSAT
          <dbl>            <dbl>
1          4.13                4
\end{verbatim}
\end{frame}

\begin{frame}[fragile]{}
\phantomsection\label{section-16}
\scriptsize

\begin{itemize}
\tightlist
\item
  \textbf{Does product and sales satisfaction differ by segment?: the
  base R way}
\end{itemize}

\tiny

\begin{Shaded}
\begin{Highlighting}[]
\NormalTok{satisfaction\_data[}\FunctionTok{c}\NormalTok{(}\StringTok{"iProdSAT"}\NormalTok{, }\StringTok{"iSalesSAT"}\NormalTok{)] }\SpecialCharTok{|\textgreater{}} 
  \FunctionTok{aggregate}\NormalTok{(}\AttributeTok{by =}\NormalTok{ satisfaction\_data[}\FunctionTok{c}\NormalTok{(}\StringTok{"Segment"}\NormalTok{)], }\AttributeTok{FUN =}\NormalTok{ mean) }\SpecialCharTok{|\textgreater{}}
  \FunctionTok{setNames}\NormalTok{(}\AttributeTok{nm =} \FunctionTok{c}\NormalTok{(}\StringTok{"Segment"}\NormalTok{, }\StringTok{"mean\_iProdSAT"}\NormalTok{, }\StringTok{"mean\_iSalesSAT"}\NormalTok{))}
\end{Highlighting}
\end{Shaded}

\begin{verbatim}
  Segment mean_iProdSAT mean_iSalesSAT
1       1      3.462963       2.981481
2       2      3.725191       3.381679
3       3      4.103896       3.811688
4       4      4.708075       4.409938
\end{verbatim}

\scriptsize

\begin{itemize}
\tightlist
\item
  \textbf{Does product and sales satisfaction differ by segment?: the
  tidyverse way}
\end{itemize}

\tiny

\begin{Shaded}
\begin{Highlighting}[]
\NormalTok{satisfaction\_data }\SpecialCharTok{|\textgreater{}}
  \FunctionTok{group\_by}\NormalTok{(Segment) }\SpecialCharTok{|\textgreater{}} 
  \FunctionTok{select}\NormalTok{(iProdSAT, iSalesSAT) }\SpecialCharTok{|\textgreater{}}
  \FunctionTok{summarise}\NormalTok{(}\AttributeTok{mean\_iProdSAT =} \FunctionTok{mean}\NormalTok{(iProdSAT), }\AttributeTok{mean\_iSalesSAT =} \FunctionTok{mean}\NormalTok{(iSalesSAT))}
\end{Highlighting}
\end{Shaded}

\begin{verbatim}
# A tibble: 4 x 3
  Segment mean_iProdSAT mean_iSalesSAT
  <fct>           <dbl>          <dbl>
1 1                3.46           2.98
2 2                3.73           3.38
3 3                4.10           3.81
4 4                4.71           4.41
\end{verbatim}
\end{frame}

\section{Acknowledgments}\label{acknowledgments}

\begin{frame}{}
\phantomsection\label{section-17}
\begin{itemize}
\item
  To my family that supports me
\item
  To the taxpayers of Colombia and the
  \href{https://www.umng.edu.co/estudiante}{\textbf{UMNG students}} who
  pay my salary
\item
  To the \href{https://www.business-science.io/}{\textbf{Business
  Science}} and \href{https://www.rfordatasci.com/}{\textbf{R4DS Online
  Learning}} communities where I learn
  \href{https://www.r-project.org/about.html}{\textbf{R}} and
  \href{https://www.python.org/about/}{\textbf{\(\pi\)-thon}}
\item
  To the \href{https://www.r-project.org/contributors.html}{\textbf{R
  Core Team}}, the creators of
  \href{https://posit.co/products/open-source/rstudio/}{\textbf{RStudio
  IDE}}, \href{https://quarto.org/}{\textbf{Quarto}} and the authors and
  maintainers of the packages
  \href{https://CRAN.R-project.org/package=tidyverse}{\textbf{tidyverse}},
  \href{https://CRAN.R-project.org/package=skimr}{\textbf{skimr}} and
  \href{https://CRAN.R-project.org/package=tinytex}{\textbf{tinytex}}
  for allowing me to access these tools without paying for a license
\item
  To the \href{https://www.kernel.org/category/about.html}{\textbf{Linux
  kernel community}} for allowing me the possibility to use some
  \href{https://static.lwn.net/Distributions/}{\textbf{Linux
  distributions}} as my main
  \href{https://en.wikipedia.org/wiki/Operating_system}{\textbf{OS}}
  without paying for a license
\end{itemize}
\end{frame}

\section*{References}\label{references}
\addcontentsline{toc}{section}{References}

\begin{frame}[allowframebreaks]{References}
\phantomsection\label{refs}
\begin{CSLReferences}{1}{0}
\bibitem[\citeproctext]{ref-chapman_r_2019}
Chapman, Chris, and Elea McDonnell Feit. 2019. \emph{R {For} {Marketing}
{Research} and {Analytics}}. 2nd ed. 2019. Use {R}! Cham: Springer
International Publishing : Imprint: Springer.
\url{https://doi-org.ezproxy.umng.edu.co/10.1007/978-3-030-14316-9}.

\end{CSLReferences}
\end{frame}




\end{document}
